\documentclass[leqno,fleqn,12pt]{article}

\usepackage{euler,beton,concrete,url,a4wide}
\usepackage[T1]{fontenc}

%% ODER: format ==         = "\mathrel{==}"
%% ODER: format /=         = "\neq "
%
%
\makeatletter
\@ifundefined{lhs2tex.lhs2tex.sty.read}%
  {\@namedef{lhs2tex.lhs2tex.sty.read}{}%
   \newcommand\SkipToFmtEnd{}%
   \newcommand\EndFmtInput{}%
   \long\def\SkipToFmtEnd#1\EndFmtInput{}%
  }\SkipToFmtEnd

\newcommand\ReadOnlyOnce[1]{\@ifundefined{#1}{\@namedef{#1}{}}\SkipToFmtEnd}
\usepackage{amstext}
\usepackage{amssymb}
\usepackage{stmaryrd}
\DeclareFontFamily{OT1}{cmtex}{}
\DeclareFontShape{OT1}{cmtex}{m}{n}
  {<5><6><7><8>cmtex8
   <9>cmtex9
   <10><10.95><12><14.4><17.28><20.74><24.88>cmtex10}{}
\DeclareFontShape{OT1}{cmtex}{m}{it}
  {<-> ssub * cmtt/m/it}{}
\newcommand{\texfamily}{\fontfamily{cmtex}\selectfont}
\DeclareFontShape{OT1}{cmtt}{bx}{n}
  {<5><6><7><8>cmtt8
   <9>cmbtt9
   <10><10.95><12><14.4><17.28><20.74><24.88>cmbtt10}{}
\DeclareFontShape{OT1}{cmtex}{bx}{n}
  {<-> ssub * cmtt/bx/n}{}
\newcommand{\tex}[1]{\text{\texfamily#1}}	% NEU

\newcommand{\Sp}{\hskip.33334em\relax}


\newcommand{\Conid}[1]{\mathit{#1}}
\newcommand{\Varid}[1]{\mathit{#1}}
\newcommand{\anonymous}{\kern0.06em \vbox{\hrule\@width.5em}}
\newcommand{\plus}{\mathbin{+\!\!\!+}}
\newcommand{\bind}{\mathbin{>\!\!\!>\mkern-6.7mu=}}
\newcommand{\rbind}{\mathbin{=\mkern-6.7mu<\!\!\!<}}% suggested by Neil Mitchell
\newcommand{\sequ}{\mathbin{>\!\!\!>}}
\renewcommand{\leq}{\leqslant}
\renewcommand{\geq}{\geqslant}
\usepackage{polytable}

%mathindent has to be defined
\@ifundefined{mathindent}%
  {\newdimen\mathindent\mathindent\leftmargini}%
  {}%

\def\resethooks{%
  \global\let\SaveRestoreHook\empty
  \global\let\ColumnHook\empty}
\newcommand*{\savecolumns}[1][default]%
  {\g@addto@macro\SaveRestoreHook{\savecolumns[#1]}}
\newcommand*{\restorecolumns}[1][default]%
  {\g@addto@macro\SaveRestoreHook{\restorecolumns[#1]}}
\newcommand*{\aligncolumn}[2]%
  {\g@addto@macro\ColumnHook{\column{#1}{#2}}}

\resethooks

\newcommand{\onelinecommentchars}{\quad-{}- }
\newcommand{\commentbeginchars}{\enskip\{-}
\newcommand{\commentendchars}{-\}\enskip}

\newcommand{\visiblecomments}{%
  \let\onelinecomment=\onelinecommentchars
  \let\commentbegin=\commentbeginchars
  \let\commentend=\commentendchars}

\newcommand{\invisiblecomments}{%
  \let\onelinecomment=\empty
  \let\commentbegin=\empty
  \let\commentend=\empty}

\visiblecomments

\newlength{\blanklineskip}
\setlength{\blanklineskip}{1mm}

\newcommand{\hsindent}[1]{\quad}% default is fixed indentation
\let\hspre\empty
\let\hspost\empty
\newcommand{\NB}{\textbf{NB}}
\newcommand{\Todo}[1]{$\langle$\textbf{To do:}~#1$\rangle$}

\EndFmtInput
\makeatother
%
%
%
%
%
%
% This package provides two environments suitable to take the place
% of hscode, called "plainhscode" and "arrayhscode". 
%
% The plain environment surrounds each code block by vertical space,
% and it uses \abovedisplayskip and \belowdisplayskip to get spacing
% similar to formulas. Note that if these dimensions are changed,
% the spacing around displayed math formulas changes as well.
% All code is indented using \leftskip.
%
% Changed 19.08.2004 to reflect changes in colorcode. Should work with
% CodeGroup.sty.
%
\ReadOnlyOnce{polycode.fmt}%
\makeatletter

\newcommand{\hsnewpar}[1]%
  {{\parskip=0pt\parindent=0pt\par\vskip #1\noindent}}

% can be used, for instance, to redefine the code size, by setting the
% command to \small or something alike
\newcommand{\hscodestyle}{}

% The command \sethscode can be used to switch the code formatting
% behaviour by mapping the hscode environment in the subst directive
% to a new LaTeX environment.

\newcommand{\sethscode}[1]%
  {\expandafter\let\expandafter\hscode\csname #1\endcsname
   \expandafter\let\expandafter\endhscode\csname end#1\endcsname}

% "compatibility" mode restores the non-polycode.fmt layout.

\newenvironment{compathscode}%
  {\par\noindent
   \advance\leftskip\mathindent
   \hscodestyle
   \let\\=\@normalcr
   \(\pboxed}%
  {\endpboxed\)%
   \par\noindent
   \ignorespacesafterend}

\newcommand{\compaths}{\sethscode{compathscode}}

% "plain" mode is the proposed default.
% It should now work with \centering.
% This required some changes. The old version
% is still available for reference as oldplainhscode.

\newenvironment{plainhscode}%
  {\hsnewpar\abovedisplayskip
   \advance\leftskip\mathindent
   \hscodestyle
   \let\hspre\(\let\hspost\)%
   \pboxed}%
  {\endpboxed%
   \hsnewpar\belowdisplayskip
   \ignorespacesafterend}

\newenvironment{oldplainhscode}%
  {\hsnewpar\abovedisplayskip
   \advance\leftskip\mathindent
   \hscodestyle
   \let\\=\@normalcr
   \(\pboxed}%
  {\endpboxed\)%
   \hsnewpar\belowdisplayskip
   \ignorespacesafterend}

% Here, we make plainhscode the default environment.

\newcommand{\plainhs}{\sethscode{plainhscode}}
\newcommand{\oldplainhs}{\sethscode{oldplainhscode}}
\plainhs

% The arrayhscode is like plain, but makes use of polytable's
% parray environment which disallows page breaks in code blocks.

\newenvironment{arrayhscode}%
  {\hsnewpar\abovedisplayskip
   \advance\leftskip\mathindent
   \hscodestyle
   \let\\=\@normalcr
   \(\parray}%
  {\endparray\)%
   \hsnewpar\belowdisplayskip
   \ignorespacesafterend}

\newcommand{\arrayhs}{\sethscode{arrayhscode}}

% The mathhscode environment also makes use of polytable's parray 
% environment. It is supposed to be used only inside math mode 
% (I used it to typeset the type rules in my thesis).

\newenvironment{mathhscode}%
  {\parray}{\endparray}

\newcommand{\mathhs}{\sethscode{mathhscode}}

% texths is similar to mathhs, but works in text mode.

\newenvironment{texthscode}%
  {\(\parray}{\endparray\)}

\newcommand{\texths}{\sethscode{texthscode}}

% The framed environment places code in a framed box.

\def\codeframewidth{\arrayrulewidth}
\RequirePackage{calc}

\newenvironment{framedhscode}%
  {\parskip=\abovedisplayskip\par\noindent
   \hscodestyle
   \arrayrulewidth=\codeframewidth
   \tabular{@{}|p{\linewidth-2\arraycolsep-2\arrayrulewidth-2pt}|@{}}%
   \hline\framedhslinecorrect\\{-1.5ex}%
   \let\endoflinesave=\\
   \let\\=\@normalcr
   \(\pboxed}%
  {\endpboxed\)%
   \framedhslinecorrect\endoflinesave{.5ex}\hline
   \endtabular
   \parskip=\belowdisplayskip\par\noindent
   \ignorespacesafterend}

\newcommand{\framedhslinecorrect}[2]%
  {#1[#2]}

\newcommand{\framedhs}{\sethscode{framedhscode}}

% The inlinehscode environment is an experimental environment
% that can be used to typeset displayed code inline.

\newenvironment{inlinehscode}%
  {\(\def\column##1##2{}%
   \let\>\undefined\let\<\undefined\let\\\undefined
   \newcommand\>[1][]{}\newcommand\<[1][]{}\newcommand\\[1][]{}%
   \def\fromto##1##2##3{##3}%
   \def\nextline{}}{\) }%

\newcommand{\inlinehs}{\sethscode{inlinehscode}}

% The joincode environment is a separate environment that
% can be used to surround and thereby connect multiple code
% blocks.

\newenvironment{joincode}%
  {\let\orighscode=\hscode
   \let\origendhscode=\endhscode
   \def\endhscode{\def\hscode{\endgroup\def\@currenvir{hscode}\\}\begingroup}
   %\let\SaveRestoreHook=\empty
   %\let\ColumnHook=\empty
   %\let\resethooks=\empty
   \orighscode\def\hscode{\endgroup\def\@currenvir{hscode}}}%
  {\origendhscode
   \global\let\hscode=\orighscode
   \global\let\endhscode=\origendhscode}%

\makeatother
\EndFmtInput
%


\def\prompt#1{\noindent$\gg$ #1}

\title{Enumerating the Elements of the Eisenstein Array}
\author{
   Roland Backhouse \\{\tt\small rcb@cs.nott.ac.uk} 
   \and 
   Jo\~ao F. Ferreira \\{\tt\small joao@joaoff.com}
}
\date{\today}

\begin{document}
\maketitle

\begin{abstract}
In \cite{jff*09:euclid-alg}, we discuss several algorithms that enumerate the elements of the Eisenstein 
Array \cite{stern1858:rationals}. In this document we show and discuss several Haskell implementations of these algorithms.
\end{abstract}


\section{The Eisenstein Array}
Given two natural numbers $m$ and $n$, Stern \cite{stern1858:rationals} describes a process 
(which he attributes to Eisenstein) of generating an infinite  sequence of rows of numbers.  
The \emph{zeroth}  row in the sequence (``nullte Entwickelungsreihe'') is the given pair of numbers:

\begin{displaymath}
m\ \ \ \ \ n\mbox{\ \ \ .}
\end{displaymath}

Subsequent rows are obtained by inserting between every pair of numbers the sum of the numbers.  Thus
the \emph{first} row is

\begin{displaymath}
m\ \ \ \ \ \ m{+}n\ \ \ \ \ \ n 
\end{displaymath}

and the \emph{second} row is

\begin{displaymath}
m\ \ \ \ \ \ 2{\times}m + n\ \ \ \ \ \ m{+}n\ \ \ \ \ \ m + 2{\times}n\ \ \ \ \ \ n~~. 
\end{displaymath}

The process of constructing such rows is repeated indefinitely.  The sequence of numbers obtained by
concatenating the individual rows in order is what is now called the \emph{Eisenstein array} and denoted by
 $\mathit{Ei\/}(m{,}n)$ (see, for example, \cite[sequence A064881]{sloane-integers}) .  
Stern refers to each occurrence of a number  in rows other than the
zeroth row as either  a 
\emph{sum element} (``Summenglied'')  or a  \emph{source element} (``Stammglied'').    The sum elements are the
newly added numbers.  For example, in the first row the
number $m{+}n$ is a sum element; in the second row the number $m{+}n$ is a  source element. 



\section{Newman's Algorithm}
An interesting question is whether Stern also documents the algorithm currently attributed to
Moshe Newman for enumerating the elements of $\mathit{Ei\/}(1{,}1)$ \cite[section 4.2.3]{jff*09:euclid-alg}. 
Newman's algorithm predicts that each triple of numbers in a given row of 
$\mathit{Ei\/}(1{,}1)$ has the form

\begin{displaymath}
a\ \ \ \ \ \ \ b\ \ \ \ \ \ \ (2{}\left\lfloor{\displaystyle\frac{a}{b}}\right\rfloor + 1){\times}b - a\ \ \ \ .
\end{displaymath}

In \cite[appendix A]{jff*08:rationals}, we have implemented Newman's algorithm as follows.
\begin{hscode}\SaveRestoreHook
\column{B}{@{}>{\hspre}l<{\hspost}@{}}%
\column{3}{@{}>{\hspre}l<{\hspost}@{}}%
\column{4}{@{}>{\hspre}l<{\hspost}@{}}%
\column{11}{@{}>{\hspre}l<{\hspost}@{}}%
\column{21}{@{}>{\hspre}c<{\hspost}@{}}%
\column{21E}{@{}l@{}}%
\column{24}{@{}>{\hspre}l<{\hspost}@{}}%
\column{29}{@{}>{\hspre}l<{\hspost}@{}}%
\column{36}{@{}>{\hspre}l<{\hspost}@{}}%
\column{E}{@{}>{\hspre}l<{\hspost}@{}}%
\>[3]{}\Varid{cwnEnum}\mathbin{::}[\mskip1.5mu \Conid{Rational}\mskip1.5mu]{}\<[E]%
\\
\>[3]{}\Varid{cwnEnum}\mathrel{=}\Varid{iterate}\;\Varid{nextCW}\ \mathrm{1}/\mathrm{1}{}\<[E]%
\\
\>[3]{}\hsindent{1}{}\<[4]%
\>[4]{}\mathbf{where}\;{}\<[11]%
\>[11]{}\Varid{nextCW}\mathbin{::}\Conid{Rational}\to \Conid{Rational}{}\<[E]%
\\
\>[11]{}\Varid{nextCW}\;\Varid{r}{}\<[21]%
\>[21]{}\mathrel{=}{}\<[21E]%
\>[24]{}\mathbf{let}\;{}\<[29]%
\>[29]{}(\Varid{n},\Varid{m}){}\<[36]%
\>[36]{}\mathrel{=}(\Varid{numerator}\;\Varid{r},\Varid{denominator}\;\Varid{r}){}\<[E]%
\\
\>[29]{}\Varid{j}{}\<[36]%
\>[36]{}\mathrel{=}\lfloor\Varid{n}/\Varid{m}\rfloor{}\<[E]%
\\
\>[24]{}\mathbf{in}\;{}\<[29]%
\>[29]{}\Varid{m}/((\mathrm{2}\mskip-4mu\times\mskip-4mu\Varid{j}\mathbin{+}\mathrm{1})\mskip-4mu\times\mskip-4mu\Varid{m}\mathbin{-}\Varid{n}){}\<[E]%
\ColumnHook
\end{hscode}\resethooks
For the purpose of this document, we are interested in the elements of $\mathit{Ei\/}(1{,}1)$, i.e.,
in the sequence of numerators given by \ensuremath{\Varid{cwnEnum}}. Function \ensuremath{\Varid{newman}} enumerates the elements of
$\mathit{Ei\/}(1{,}1)$, using the infinite list created by \ensuremath{\Varid{cwnEnum}} (we have to detect a change in level).
\begin{hscode}\SaveRestoreHook
\column{B}{@{}>{\hspre}l<{\hspost}@{}}%
\column{3}{@{}>{\hspre}l<{\hspost}@{}}%
\column{5}{@{}>{\hspre}l<{\hspost}@{}}%
\column{21}{@{}>{\hspre}l<{\hspost}@{}}%
\column{45}{@{}>{\hspre}l<{\hspost}@{}}%
\column{E}{@{}>{\hspre}l<{\hspost}@{}}%
\>[3]{}\Varid{newman}\mathbin{::}[\mskip1.5mu \Conid{Integer}\mskip1.5mu]{}\<[E]%
\\
\>[3]{}\Varid{newman}\mathrel{=}\Varid{concatMap}\;\Varid{dlevel}\;\Varid{cwnEnum}{}\<[E]%
\\
\>[3]{}\hsindent{2}{}\<[5]%
\>[5]{}\mathbf{where}\;\Varid{dlevel}\;\Varid{r}{}\<[21]%
\>[21]{}\mid (\Varid{denominator}\;\Varid{r})==\mathrm{1}{}\<[45]%
\>[45]{}\mathrel{=}[\mskip1.5mu \Varid{numerator}\;\Varid{r},\mathrm{1}\mskip1.5mu]{}\<[E]%
\\
\>[21]{}\mid \Varid{otherwise}{}\<[45]%
\>[45]{}\mathrel{=}[\mskip1.5mu \Varid{numerator}\;\Varid{r}\mskip1.5mu]{}\<[E]%
\ColumnHook
\end{hscode}\resethooks
\section{Enumerating the Elements of $\mathit{Ei\/}(m{,}n)$}
One way of enumerating the elements of the array $\mathit{Ei\/}(m{,}n)$ is:
\begin{hscode}\SaveRestoreHook
\column{B}{@{}>{\hspre}l<{\hspost}@{}}%
\column{3}{@{}>{\hspre}l<{\hspost}@{}}%
\column{4}{@{}>{\hspre}l<{\hspost}@{}}%
\column{11}{@{}>{\hspre}l<{\hspost}@{}}%
\column{35}{@{}>{\hspre}l<{\hspost}@{}}%
\column{39}{@{}>{\hspre}l<{\hspost}@{}}%
\column{E}{@{}>{\hspre}l<{\hspost}@{}}%
\>[3]{}\Varid{ei}\mathbin{::}\Conid{Integer}\to \Conid{Integer}\to [\mskip1.5mu \Conid{Integer}\mskip1.5mu]{}\<[E]%
\\
\>[3]{}\Varid{ei}\;\Varid{m}\;\Varid{n}\mathrel{=}\Varid{m}\mathbin{:}\Varid{eiloop}\;\mathrm{1}\;\mathrm{1}\;\Varid{m}\;\Varid{n}\;\Varid{m}\;\Varid{n}{}\<[E]%
\\
\>[3]{}\hsindent{1}{}\<[4]%
\>[4]{}\mathbf{where}\;{}\<[11]%
\>[11]{}\Varid{eiloop}\;\Varid{a}\;\mathrm{1}\;\Varid{m}\;\Varid{n}\;\Varid{cm}\;\Varid{cn}\mathrel{=}{}\<[35]%
\>[35]{}\Varid{n}\mathbin{:}\Varid{cm}\mathbin{:}\Varid{eiloop}\;\mathrm{1}\;(\Varid{a}\mathbin{+}\mathrm{1})\;\Varid{cm}\;(\Varid{a}\mskip-4mu\times\mskip-4mu\Varid{cm}\mathbin{+}\Varid{cn})\;\Varid{cm}\;\Varid{cn}{}\<[E]%
\\
\>[11]{}\Varid{eiloop}\;\Varid{a}\;\Varid{b}\;\Varid{m}\;\Varid{n}\;\Varid{cm}\;\Varid{cn}\mathrel{=}{}\<[35]%
\>[35]{}\mathbf{let}\;\Varid{k}\mathrel{=}\mathrm{2}\mskip-4mu\times\mskip-4mu\lfloor\Varid{a}/\Varid{b}\rfloor\mathbin{+}\mathrm{1}{}\<[E]%
\\
\>[35]{}\mathbf{in}\;{}\<[39]%
\>[39]{}\Varid{n}\mathbin{:}\Varid{eiloop}\;\Varid{b}\;(\Varid{k}\mskip-4mu\times\mskip-4mu\Varid{b}\mathbin{-}\Varid{a})\;\Varid{n}\;(\Varid{k}\mskip-4mu\times\mskip-4mu\Varid{n}\mathbin{-}\Varid{m})\;\Varid{cm}\;\Varid{cn}{}\<[E]%
\ColumnHook
\end{hscode}\resethooks
We can test if the function \ensuremath{\Varid{newman}} is in fact enumerating the elements of $\mathit{Ei\/}(1{,}1)$.
Let's compare the first $1000$ elements of both enumerations:

\medskip
\prompt{(take 1000 newman) == (take 1000 (ei 1 1))}\\
\ensuremath{\Conid{True}}\\
\prompt{(take 1000 (map (2*) newman)) == (take 1000 (ei 2 2))}\\
\ensuremath{\Conid{True}}\\
%\prompt{(take 1000 (map ((-2)*) newman)) == (take 1000 (ei (-2) (-2)))}\\
%\eval{(take 1000 (map ((-2)*) newman)) == (take 1000 (ei (-2) (-2)))}\\

The part after the prompt, $\gg$, is the Haskell code that \texttt{ghci} is executing. The result is shown in
the subsequent line. The second command shows an instance of the property:

\[
\ensuremath{\Varid{map}\;(\Varid{k}\mskip-4mu\times\mskip-4mu)\;(\Varid{ei}\;\mathrm{1}\;\mathrm{1})==\Varid{ei}\;\Varid{k}\;\Varid{k}} ~~~~~.
\]

%TODO: Conclude this part
A property that we have not yet proved is that we can replace \ensuremath{\Varid{a}} and \ensuremath{\Varid{b}} by \ensuremath{\Varid{m}} and \ensuremath{\Varid{n}} in the calculation of \ensuremath{\Varid{k}} (when \ensuremath{\Varid{m}} and \ensuremath{\Varid{n}} are both positive).
\begin{hscode}\SaveRestoreHook
\column{B}{@{}>{\hspre}l<{\hspost}@{}}%
\column{3}{@{}>{\hspre}l<{\hspost}@{}}%
\column{4}{@{}>{\hspre}l<{\hspost}@{}}%
\column{11}{@{}>{\hspre}l<{\hspost}@{}}%
\column{35}{@{}>{\hspre}l<{\hspost}@{}}%
\column{39}{@{}>{\hspre}l<{\hspost}@{}}%
\column{E}{@{}>{\hspre}l<{\hspost}@{}}%
\>[3]{}\Varid{ei'}\mathbin{::}\Conid{Integer}\to \Conid{Integer}\to [\mskip1.5mu \Conid{Integer}\mskip1.5mu]{}\<[E]%
\\
\>[3]{}\Varid{ei'}\;\Varid{m}\;\Varid{n}\mathrel{=}\Varid{m}\mathbin{:}\Varid{eiloop}\;\mathrm{1}\;\mathrm{1}\;\Varid{m}\;\Varid{n}\;\Varid{m}\;\Varid{n}{}\<[E]%
\\
\>[3]{}\hsindent{1}{}\<[4]%
\>[4]{}\mathbf{where}\;{}\<[11]%
\>[11]{}\Varid{eiloop}\;\Varid{a}\;\mathrm{1}\;\Varid{m}\;\Varid{n}\;\Varid{cm}\;\Varid{cn}\mathrel{=}{}\<[35]%
\>[35]{}\Varid{n}\mathbin{:}\Varid{cm}\mathbin{:}\Varid{eiloop}\;\mathrm{1}\;(\Varid{a}\mathbin{+}\mathrm{1})\;\Varid{cm}\;(\Varid{a}\mskip-4mu\times\mskip-4mu\Varid{cm}\mathbin{+}\Varid{cn})\;\Varid{cm}\;\Varid{cn}{}\<[E]%
\\
\>[11]{}\Varid{eiloop}\;\Varid{a}\;\Varid{b}\;\Varid{m}\;\Varid{n}\;\Varid{cm}\;\Varid{cn}\mathrel{=}{}\<[35]%
\>[35]{}\mathbf{let}\;\Varid{k}\mathrel{=}\mathrm{2}\mskip-4mu\times\mskip-4mu\lfloor\Varid{m}/\Varid{n}\rfloor\mathbin{+}\mathrm{1}{}\<[E]%
\\
\>[35]{}\mathbf{in}\;{}\<[39]%
\>[39]{}\Varid{n}\mathbin{:}\Varid{eiloop}\;\Varid{b}\;(\Varid{k}\mskip-4mu\times\mskip-4mu\Varid{b}\mathbin{-}\Varid{a})\;\Varid{n}\;(\Varid{k}\mskip-4mu\times\mskip-4mu\Varid{n}\mathbin{-}\Varid{m})\;\Varid{cm}\;\Varid{cn}{}\<[E]%
\ColumnHook
\end{hscode}\resethooks
We now define the function \ensuremath{\Varid{test}}, which compares the first $1000$ elements of two enumerations of $\mathit{Ei\/}(m{,}n)$, with $0{\leq}m{\leq}x$ and $1{\leq}n{\leq}x$ :
\begin{hscode}\SaveRestoreHook
\column{B}{@{}>{\hspre}l<{\hspost}@{}}%
\column{3}{@{}>{\hspre}l<{\hspost}@{}}%
\column{E}{@{}>{\hspre}l<{\hspost}@{}}%
\>[3]{}\Varid{test}\;\Varid{f}\;\Varid{g}\;\Varid{x}\mathrel{=}\Varid{and}\;[\mskip1.5mu \Varid{take}\;\mathrm{1000}\;(\Varid{f}\;\Varid{m}\;\Varid{n})==\Varid{take}\;\mathrm{1000}\;(\Varid{g}\;\Varid{m}\;\Varid{n})\mid \Varid{m}\leftarrow [\mskip1.5mu \mathrm{0}\mathinner{\ldotp\ldotp}\Varid{x}\mskip1.5mu],\Varid{n}\leftarrow [\mskip1.5mu \mathrm{1}\mathinner{\ldotp\ldotp}\Varid{x}\mskip1.5mu]\mskip1.5mu]{}\<[E]%
\ColumnHook
\end{hscode}\resethooks
We can use \ensuremath{\Varid{test}} to see if the first $1000$ elements of \ensuremath{\Varid{ei}\;\Varid{m}\;\Varid{n}} and \ensuremath{\Varid{ei'}\;\Varid{m}\;\Varid{n}},
for $0{\leq}m{\leq}100$ and $1{\leq}n{\leq}100$, are the same (we are testing $10100$ pairs).

\medskip
\prompt{test ei ei' 100}\\
\ensuremath{\Conid{True}}\\ %eval takes too long! But I've checked it (02/03/2009)
%\eval{test ei ei' 100}

The function \ensuremath{\Varid{extnewman}}, defined below, is the same as \ensuremath{\Varid{ei}}, but it replaces variables \ensuremath{\Varid{a}} and \ensuremath{\Varid{b}} by
variable \ensuremath{\Varid{r}}:
\begin{hscode}\SaveRestoreHook
\column{B}{@{}>{\hspre}l<{\hspost}@{}}%
\column{3}{@{}>{\hspre}l<{\hspost}@{}}%
\column{4}{@{}>{\hspre}l<{\hspost}@{}}%
\column{11}{@{}>{\hspre}l<{\hspost}@{}}%
\column{29}{@{}>{\hspre}l<{\hspost}@{}}%
\column{44}{@{}>{\hspre}l<{\hspost}@{}}%
\column{48}{@{}>{\hspre}l<{\hspost}@{}}%
\column{E}{@{}>{\hspre}l<{\hspost}@{}}%
\>[3]{}\Varid{extnewman}\mathbin{::}\Conid{Integer}\to \Conid{Integer}\to [\mskip1.5mu \Conid{Integer}\mskip1.5mu]{}\<[E]%
\\
\>[3]{}\Varid{extnewman}\;\Varid{cm}\;\Varid{cn}\mathrel{=}\Varid{cm}\mathbin{:}\Varid{loop}\;\mathrm{0}\;\Varid{cm}\;\Varid{cn}\;\Varid{cm}\;\Varid{cn}{}\<[E]%
\\
\>[3]{}\hsindent{1}{}\<[4]%
\>[4]{}\mathbf{where}\;{}\<[11]%
\>[11]{}\Varid{loop}\;\Varid{r}\;\Varid{m}\;\Varid{n}\;\Varid{cm}\;\Varid{cn}{}\<[29]%
\>[29]{}\mid ((\Varid{m}==(\Varid{cm}\mathbin{+}\Varid{r}\mskip-4mu\times\mskip-4mu\Varid{cn}))\mathrel{\wedge}(\Varid{n}==\Varid{cn}))\mathrel{=}{}\<[E]%
\\
\>[29]{}\hsindent{15}{}\<[44]%
\>[44]{}\Varid{n}\mathbin{:}\Varid{cm}\mathbin{:}\Varid{loop}\;(\Varid{r}\mathbin{+}\mathrm{1})\;\Varid{cm}\;((\Varid{r}\mathbin{+}\mathrm{1})\mskip-4mu\times\mskip-4mu\Varid{cm}\mathbin{+}\Varid{cn})\;\Varid{cm}\;\Varid{cn}{}\<[E]%
\\
\>[29]{}\mid \Varid{otherwise}\mathrel{=}{}\<[44]%
\>[44]{}\mathbf{let}\;\Varid{k}\mathrel{=}\mathrm{2}\mskip-4mu\times\mskip-4mu\lfloor\Varid{m}/\Varid{n}\rfloor\mathbin{+}\mathrm{1}{}\<[E]%
\\
\>[44]{}\mathbf{in}\;{}\<[48]%
\>[48]{}\Varid{n}\mathbin{:}\Varid{loop}\;\Varid{r}\;\Varid{n}\;(\Varid{k}\mskip-4mu\times\mskip-4mu\Varid{n}\mathbin{-}\Varid{m})\;\Varid{cm}\;\Varid{cn}{}\<[E]%
\ColumnHook
\end{hscode}\resethooks
We can do a similar test for \ensuremath{\Varid{extnewman}} as we did for \ensuremath{\Varid{ei'}}:

\medskip
\prompt{test ei extnewman 100}\\
\ensuremath{\Conid{True}} %eval takes too long! But I've checked it (02/03/2009)
%\eval{test ei extnewman 100}

\section{The Online Encyclopedia of Integer Sequences}
In this section, we show how we can use the functions from the Haskell module \ensuremath{\Conid{\Conid{Math}.OEIS}}\footnote{To run this literate haskell file, 
you need to have the module \ensuremath{\Conid{\Conid{Math}.OEIS}} installed. You can download it at \url{http://hackage.haskell.org/cgi-bin/hackage-scripts/package/oeis}.} 
to search for occurrences of the Eisenstein array on the Online Encyclopedia of Integer Sequences (OEIS) \cite{sloane-integers}.

We start by defining the number of elements, \ensuremath{\Varid{numElems}}, that we want to send to the OEIS, and a function that
converts a list $[\,x_1,\cdots,x_n\,]$ to the string $"\,x_1,\cdots,x_n\,"$:
\begin{hscode}\SaveRestoreHook
\column{B}{@{}>{\hspre}l<{\hspost}@{}}%
\column{3}{@{}>{\hspre}l<{\hspost}@{}}%
\column{13}{@{}>{\hspre}l<{\hspost}@{}}%
\column{E}{@{}>{\hspre}l<{\hspost}@{}}%
\>[3]{}\Varid{numElems}{}\<[13]%
\>[13]{}\mathbin{::}\Conid{Int}{}\<[E]%
\\
\>[3]{}\Varid{numElems}{}\<[13]%
\>[13]{}\mathrel{=}\mathrm{20}{}\<[E]%
\ColumnHook
\end{hscode}\resethooks
\begin{hscode}\SaveRestoreHook
\column{B}{@{}>{\hspre}l<{\hspost}@{}}%
\column{3}{@{}>{\hspre}l<{\hspost}@{}}%
\column{16}{@{}>{\hspre}l<{\hspost}@{}}%
\column{19}{@{}>{\hspre}l<{\hspost}@{}}%
\column{E}{@{}>{\hspre}l<{\hspost}@{}}%
\>[3]{}\Varid{list2string}{}\<[16]%
\>[16]{}\mathbin{::}(\Conid{Show}\;\Varid{a})\Rightarrow [\mskip1.5mu \Varid{a}\mskip1.5mu]\to \Conid{String}{}\<[E]%
\\
\>[3]{}\Varid{list2string}{}\<[16]%
\>[16]{}\mathrel{=}{}\<[19]%
\>[19]{}\Varid{init}\mathbin{\circ}\Varid{tail}\mathbin{\circ}\Varid{show}{}\<[E]%
\ColumnHook
\end{hscode}\resethooks
The following function, \ensuremath{\Varid{oeis}}, receives two integer numbers, \ensuremath{\Varid{m}} and \ensuremath{\Varid{n}}, computes the list of the first \ensuremath{\Varid{numElems}} of \ensuremath{\Varid{ei}\;\Varid{m}\;\Varid{n}},
transforms it into a string and checks if it exists in the OEIS. It prints the description of the sequence, together
with its reference.
\begin{hscode}\SaveRestoreHook
\column{B}{@{}>{\hspre}l<{\hspost}@{}}%
\column{3}{@{}>{\hspre}l<{\hspost}@{}}%
\column{5}{@{}>{\hspre}l<{\hspost}@{}}%
\column{12}{@{}>{\hspre}l<{\hspost}@{}}%
\column{13}{@{}>{\hspre}l<{\hspost}@{}}%
\column{16}{@{}>{\hspre}l<{\hspost}@{}}%
\column{20}{@{}>{\hspre}l<{\hspost}@{}}%
\column{35}{@{}>{\hspre}l<{\hspost}@{}}%
\column{38}{@{}>{\hspre}l<{\hspost}@{}}%
\column{E}{@{}>{\hspre}l<{\hspost}@{}}%
\>[3]{}\Varid{oeis}{}\<[13]%
\>[13]{}\mathbin{::}\Conid{Integer}\to \Conid{Integer}\to \Conid{IO}\;(){}\<[E]%
\\
\>[3]{}\Varid{oeis}\;\Varid{m}\;\Varid{n}{}\<[13]%
\>[13]{}\mathrel{=}{}\<[16]%
\>[16]{}\mathbf{do}\;{}\<[20]%
\>[20]{}\Varid{s}\leftarrow \Varid{searchSequence\char95 IO}\mathbin{\circ}\Varid{list2string}\mathbin{\circ}(\Varid{take}\;\Varid{numElems})\ \Varid{ei}\;\Varid{m}\;\Varid{n}{}\<[E]%
\\
\>[20]{}\Varid{r}\leftarrow \Varid{getDataSeq}\;\Varid{s}{}\<[E]%
\\
\>[20]{}\Varid{putStrLn}\ \text{\tt \char34 Ei(\char34}\plus \Varid{show}\;\Varid{m}\plus \text{\tt \char34 ,\char34}\plus \Varid{show}\;\Varid{n}\plus \text{\tt \char34 ):\char92 n\char92 t\char34}\plus \Varid{r}{}\<[E]%
\\
\>[3]{}\hsindent{2}{}\<[5]%
\>[5]{}\mathbf{where}\;{}\<[12]%
\>[12]{}\Varid{getDataSeq}{}\<[35]%
\>[35]{}\mathbin{::}(\Conid{Maybe}\;\Conid{OEISSequence})\to (\Conid{IO}\;\Conid{String}){}\<[E]%
\\
\>[12]{}\Varid{getDataSeq}\;\Conid{Nothing}{}\<[35]%
\>[35]{}\mathrel{=}{}\<[38]%
\>[38]{}\Varid{return}\;\text{\tt \char34 Sequence~not~found.\char34}{}\<[E]%
\\
\>[12]{}\Varid{getDataSeq}\;(\Conid{Just}\;\Varid{seq}){}\<[35]%
\>[35]{}\mathrel{=}{}\<[38]%
\>[38]{}\Varid{return}\ (\Varid{description}\;\Varid{seq})\plus \text{\tt \char34 ~(~\char34}\plus (\Varid{concatMap}\;(\plus \text{\tt \char34 ~\char34})\;(\Varid{catalogNums}\;\Varid{seq}))\plus \text{\tt \char34 )\char34}{}\<[E]%
\ColumnHook
\end{hscode}\resethooks
As an example, here is the output for the sequence \ensuremath{\Varid{ei}\;\mathrm{1}\;\mathrm{1}}:

\medskip
\prompt{oeis 1 1}
\noindent\begin{tabbing}\tt
~Ei\char40{}1\char44{}1\char41{}\char58{}\\
\tt ~~~~~~~~~Triangle~T\char40{}n\char44{}k\char41{}~\char61{}~denominator~of~fraction~in~k\char45{}th~term~of~n\char45{}th~row~of\\
\tt ~variant~of~Farey~series\char46{}~This~is~also~Stern\char39{}s~diatomic~array~read~by\\
\tt ~rows~\char40{}version~1\char41{}\char46{}~\char40{}~A049456~\char41{}
\end{tabbing}
%\eval{oeis 1 1}



\bibliographystyle{plain}
\bibliography{math,jff}
\end{document}
